% !TeX spellcheck = fa_FA
% !TEX TS-program = xelatex
% !TEX encoding = UTF-8 Unicode

\documentclass[xcolor=dvipsnames, professionalfonts, aspectratio=169, 11pt]{beamer}

\usepackage{multicol}
\usepackage{tikz}
\usetikzlibrary{automata, positioning, arrows}
\usepackage{dblfnote}
\usepackage{float}
\usepackage{setspace}

%\hypersetup{colorlinks = true}

% ---------------------------------------
% color
% ---------------------------------------

\usefonttheme{serif}
\usetheme[compress]{Ilmenau} 
% \usetheme{Madrid}
%% hide subsection row in headline in Ilmenau
\makeatletter
 \beamer@theme@subsectionfalse
\makeatother
% \usetheme{AnnArbor}
\usetheme{Frankfurt}
% \usetheme{CambridgeUs}

% \usecolortheme{wolverine}

% cool and blue
% \usecolortheme{seahorse}

% \usecolortheme{rose}

%% Warm and red theme
% \usecolortheme{beaver}
\setbeamercolor{section number projected}{bg=blue,fg=white}
\setbeamercolor*{item}{fg=blue}

%% green theme
% \usecolortheme{spruce}
% \setbeamercolor{section number projected}{bg=OliveGreen,fg=white}
% \setbeamercolor*{item}{fg=OliveGreen}

% \usecolortheme[named=green]{structure}
%\setbeamercolor{normal text}{fg=blue}\usebeamercolor*{normal text}


\usepackage{textpos}
\usepackage{makebox}
\usepackage{inputenc}
\usepackage{xcolor}
\usepackage{hyperref}
\usepackage{caption}
\usepackage{subfig}
\captionsetup{justification=centering,singlelinecheck=false} %raggedleft
%\captionsetup[table]{skip=-1cm}
\usepackage{xecolor}
\usepackage{graphicx}
\usepackage{algorithm}
\usepackage{algorithmic}

\usepackage{amsthm,amssymb,amsmath}

\usepackage{appendixnumberbeamer}

\PassOptionsToPackage{unicode}{hyperref}
% \PassOptionsToPackage{table}{xcolor}
\usepackage{colortbl}

\graphicspath{{./img/}}

\usepackage[
  extrafootnotefeatures,
  localise=on,
  mathdigits=default,
  inlinemathdigits=default,
  displaymathdigits=default % persian
]{xepersian}

%\setbeamertemplate{bibliography item}{\insertbiblabel}

% ---------------------------------------
% font
% ---------------------------------------

% تعریف قلم فارسی و انگلیسی و مکان قلم‌ها
\settextfont{XB Niloofar}


% \setbeamertemplate{section in toc}[circle]
% \setbeamertemplate{frametitle continuation}{\gdef\beamer@frametitle{}} % framebreaks without numbering
\setbeamertemplate{frametitle}[default][center]% align the frametitle to the center
\setbeamertemplate{caption}[numbered]{}% Number float-like environments
\setbeamertemplate{itemize item}{\scriptsize\raise1.25pt \hbox{\donotcoloroutermaths$\blacktriangleleft$}} % Correct the bullet for RTL texts
\setbeamertemplate{enumerate item}[circle] % ball or circle

%\setbeamersize{text margin left=5mm,text margin right=10mm}


%%%%%%%%%%%%%%%%%%%%%%%%%%%%%%%%%%%%
%%% تصحیح  \subsection و \subsubsection در فهرست مطالب
\makeatletter
\expandafter\let\csname beamer@@tmpop@subsection in toc@default\endcsname\relax
\expandafter\let\csname beamer@@tmpop@subsubsection in toc@default\endcsname\relax
\defbeamertemplate*{subsection in toc}{default}
{\leavevmode\rightskip=1.5em\inserttocsubsection\par}

\defbeamertemplate*{subsubsection in toc}{default}
{\leavevmode\normalsize\usebeamerfont{subsection in toc}\rightskip=3em%
  \usebeamerfont{subsubsection in toc}\inserttocsubsubsection\par}
\makeatother
%%%%%%%%%%%%%%%%%%%%%%%%%%%%%%%

\makeatletter
\defbeamertemplate{section in toc}{RtlTocBall}{
  \leavevmode\leftskip=2.75ex
  \llap{
    \normalsize
    \begin{pgfpicture}{-1ex}{-0.7ex}{1ex}{1ex}
      \pgftext{\beamer@usesphere{section number projected}{tocsphere}}
      \pgftext{
        \usebeamerfont*{section number projected}
        \usebeamercolor{section number projected}
        \color{fg!90!bg}
        \hspace{-0.8em}
        \inserttocsectionnumber}
    \end{pgfpicture}
    \kern1.25ex}
  \raggedleft \inserttocsection\par
}
[action]
{\setbeamerfont{section number projected}{size=\scriptsize}}

\defbeamertemplate{subsection in toc}{RtlTocBall}
{\leavevmode\leftskip=5ex
  \hspace{1.5em}\llap{\raise0.1ex\beamer@usesphere{subsection number projected}{bigsphere}\kern1ex}
  \raggedleft \inserttocsubsection\par
}

\setbeamertemplate{subsubsection in toc}{
\setRTL \rightskip=3ex\myitem
  \inserttocsubsection\par
}

\setbeamertemplate{sections/subsections in toc}[RtlTocBall]
\makeatother

\makeatletter
\newcommand{\RTList}{\raggedleft\rightskip\@totalleftmargin}
\newenvironment{shomarei}[1][]{\begin{enumerate}[#1]\RTList}{\end{enumerate}}
\newenvironment{moredi}[1][]{\begin{itemize}[#1]\RTList}{\end{itemize}}
\makeatother

%%% تصحیح دستورات \frametitle و \framesubtitle برای قرار دادن عنوان و زیرعنوان یک frame استفاده کنیم:
\makeatletter
\define@key{beamercolbox}{left}[0pt]{\def\beamer@colbox@rs{0pt}\def\beamer@colbox@ls{#1 plus1fill}}
\makeatother

%%% تصحیح محیط‌های لیست (سطوح لیست به درستی چیده نمی‌شوند)
\makeatletter
% \expandafter\let\csname beamer@@tmpop@itemize item@default\endcsname\relax
% \expandafter\let\csname beamer@@tmpop@itemize subitem@default\endcsname\relax
% \expandafter\let\csname beamer@@tmpop@itemize subsubitem@default\endcsname\relax

% \defbeamertemplate*{itemize item}{default}{\scriptsize\raise1.25pt\hbox{\donotcoloroutermaths$\blacktriangleleft$}}
% \defbeamertemplate*{itemize subitem}{default}{\tiny\raise1.5pt\hbox{\donotcoloroutermaths$\blacktriangleleft$}}
% \defbeamertemplate*{itemize subsubitem}{default}{\tiny\raise1.5pt\hbox{\donotcoloroutermaths$\blacktriangleleft$}}

% \bidi@patchcmd{\@listi}{\leftmargin}{\rightmargin}{}{}
% \let\@listI\@listi
% \bidi@patchcmd{\@listii}{\leftmargin}{\rightmargin}{}{}
% \bidi@patchcmd{\@listiii}{\leftmargin}{\rightmargin}{}{}
% \bidi@patchcmd{\beamer@enum@}{\raggedright}{\raggedleft}{}{}
% \bidi@patchcmd{\@@description}{\raggedright}{\raggedleft}{}{}
% \bidi@patchcmd{\@@description}{\leftmargin}{\rightmargin}{}{}

% \renewcommand{\itemize}[1][]{%
%   \beamer@ifempty{#1}{}{\def\beamer@defaultospec{#1}}%
%   \ifnum \@itemdepth >2\relax\@toodeep\else
%     \advance\@itemdepth\@ne
%     \beamer@computepref\@itemdepth% sets \beameritemnestingprefix
%     \usebeamerfont{itemize/enumerate \beameritemnestingprefix body}%
%     \usebeamercolor[fg]{itemize/enumerate \beameritemnestingprefix body}%
%     \usebeamertemplate{itemize/enumerate \beameritemnestingprefix body begin}%
%     \list
%       {\usebeamertemplate{itemize \beameritemnestingprefix item}}
%       {\def\makelabel##1{%
%           {%
%             \hss\llap{{%
%                 \usebeamerfont*{itemize \beameritemnestingprefix item}%
%                 \usebeamercolor[fg]{itemize \beameritemnestingprefix item}##1}}%
%           }%
%         }%
%       }
%   \fi%
%   \beamer@cramped%
%   \raggedleft%
%   \beamer@firstlineitemizeunskip%
% }
\makeatother
%%%%%%%%%%%%%%%%%%%%%%%%%%%%%%%%%%%%%%

%%% تصحیح مشکل پانویس footnote

\makeatletter
\bidi@undef\beamer@@tmpop@footnote@default

\defbeamertemplate*{footnote}{default}
{
  \parindent 1em\noindent%
  \raggedleft
  \hbox to 1.8em{\hfil\insertfootnotemark}\insertfootnotetext\par%
}

\defbeamertemplate*{LTRfootnote}{default}
{
  \parindent 1em\noindent%
  \raggedright
  \hbox to 1.8em{\hfil\insertfootnotemark}\latinfont\insertfootnotetext\par%
}
\footdir@temp\footdir@ORG@bidi@beamer@framefootnotetext\beamer@framefootnotetext{R}
\let\@footnotetext=\beamer@framefootnotetext
\let\@RTLfootnotetext\@footnotetext

\def\@makeLTRfntext#1{%
  \def\insertfootnotetext{#1}%
  \def\insertfootnotemark{\@makefnmark}%
  \usebeamertemplate***{LTRfootnote}}

\newcommand<>\beamer@frameLTRfootnotetext[1]{%
  \global\setbox\beamer@footins\vbox{\@RTLfalse%
    \hsize\framewidth
    \textwidth\hsize
    \columnwidth\hsize
    \unvbox\beamer@footins
    \reset@font\footnotesize
    \@parboxrestore
    \protected@edef\@currentlabel
    {\csname p@footnote\endcsname\@thefnmark}%
    \color@begingroup
    \uncover#2{\@makeLTRfntext{%
        \rule\z@\footnotesep\ignorespaces#1\@finalstrut\strutbox}}%
    \color@endgroup}}


\footdir@temp\footdir@ORG@bidi@beamer@frameLTRfootnotetext\beamer@frameLTRfootnotetext{L}
\let\@LTRfootnotetext=\beamer@frameLTRfootnotetext

\makeatother

%%%%%%%%%%%%%%%%%%%%%%%%%%%%%%%%%%%

%%%تصحیح چپ‌چین شدن متن
\raggedleft

% To adjust the paragraphs in RTL
\everypar{\rightskip\rightmargin}

%%%%%%%%%%%%%%%%%%%%%%%%%%


%%%تصحیح محیط‌های قضیه، مثال و ....

% \makeatletter
% \renewenvironment{beamercolorbox}[2][]{%
%   \begingroup%
%     \def\beamer@colbox@coladd{0pt}%
%     \def\beamer@vmode{\leavevmode}%
%     \setkeys{beamercolbox}{%
%       wd=\textwidth,ht={},dp={},%
%       rightskip=0pt,leftskip=0pt plus1fil,%
%       sep=0pt,colsep=0pt,colsep*=0pt,%
%       shadow=false,rounded=false,ignorebg=false}%
%     \setkeys{beamercolbox}{#1}%
%     \ifbeamercolorempty[bg]{#2}{\@tempswafalse}{\@tempswatrue}%
%     \ifbeamer@colbox@ignorebg\@tempswafalse\fi%
%     \def\beamer@colbox@color{#2}%
%     \hsize=\beamer@colbox@wd%
%     \setbox\beamer@tempbox=\hbox\bgroup\vbox\bgroup%
%       \leftskip=\beamer@colbox@ls%
%       \advance\leftskip by\beamer@colbox@sep%
%       \rightskip=\beamer@colbox@rs%
%       \advance\rightskip by\beamer@colbox@sep%
%       \ifbeamer@colbox@ignorebg%
%         \colorlet{beamer@temp@color}{bg}%
%         \usebeamercolor[fg]{#2}%
%         \colorlet{bg}{beamer@temp@color}%
%       \else%
%         \usebeamercolor[fg]{#2}%
%       \fi%
%       \if@tempswa%
%         \advance\leftskip by\beamer@colbox@colsep%
%         \advance\rightskip by\beamer@colbox@colsep%
%         \ifdim\beamer@colbox@colsep=0pt\else\vskip\beamer@colbox@colsep\fi%
%         \ifdim\beamer@colbox@colseps=0pt\else\vskip\beamer@colbox@colseps\fi%
%       \fi%
%       \ifdim\beamer@colbox@sep=0pt\else\vskip\beamer@colbox@sep\fi%
%       \beamer@vmode\ignorespaces}{%
%       \ifdim\beamer@colbox@sep=0pt\else\vskip\beamer@colbox@sep\fi%
%       \if@tempswa\ifdim\beamer@colbox@colsep=0pt\else\vskip\beamer@colbox@colsep\fi\fi%
%       \if@tempswa\ifdim\beamer@colbox@colseps=0pt\else\vskip\beamer@colbox@colseps\fi\fi%
%     \egroup\egroup%
%     \wd\beamer@tempbox=\hsize%
%     \@tempdima=\wd\beamer@tempbox%
%     \ifx\beamer@colbox@ht\@empty%
%     \else%
%       \ht\beamer@tempbox=\beamer@colbox@ht%
%     \fi%
%     \ifx\beamer@colbox@dp\@empty%
%     \else%
%       \dp\beamer@tempbox=\beamer@colbox@dp%
%     \fi%
%     \ifbeamer@colbox@rounded%
%       \if@tempswa%
%         \begin{beamerboxesrounded}[%
%           shadow=\beamer@colbox@shadow,%
%           lower=\beamer@colbox@color,%
%           upper=normal text,%
%           width=\beamer@colbox@wd]{}%
%           \box\beamer@tempbox%
%         \end{beamerboxesrounded}%
%       \else%
%         \ifdim\@tempdima>\textwidth%
%           \setbox\beamer@tempbox=\hbox to\textwidth{\hss\box\beamer@tempbox\hss}%
%         \fi%
%         \box\beamer@tempbox%
%       \fi%
%     \else%
%       \if@tempswa\setbox\beamer@tempbox=\hbox{\vbox{%
%         \usebeamercolor{\beamer@colbox@color}%
%         \advance\hsize by \beamer@colbox@colseps\relax%
%         \advance\hsize by \beamer@colbox@colseps\relax%
%         \hskip-\beamer@colbox@colseps%
%         \fboxsep=0pt\colorbox{bg}{%
%           \hskip\beamer@colbox@colseps%
%           \hbox{\box\beamer@tempbox}%
%           \hskip\beamer@colbox@colseps%
%         }%
%         \hskip-\beamer@colbox@colseps%
%       }}\fi%
%       \ifdim\@tempdima>\textwidth%
%         \setbox\beamer@tempbox=\hbox to\textwidth{\hskip0pt minus\beamer@leftmargin\relax\box\beamer@tempbox\hskip0pt minus\beamer@rightmargin\relax}%
%       \fi%
%       \box\beamer@tempbox%
%     \fi%
%   \endgroup%
% }
% \makeatother

\providetranslation{Theorem}{قضیه}
\providetranslation{Definition}{تعریف}
\providetranslation{Example}{مثال}
\renewcommand{\bibname}{مراجع}

%%%%%%%%%%%%%%%%%%%%


%%% تصحیح  ستون‌ها که چپ به راست است.

\makeatletter
\long\def\beamer@newenvnoopt#1#2#3#4{%
  \expandafter\renewcommand\expandafter<\expandafter>\csname#1\endcsname[#2]{#3}%<- here
  \expandafter\long\expandafter\def\csname end#1\endcsname{#4}%
}
\long\def\beamer@newenvopt#1#2[#3]#4#5{%
  \expandafter\renewcommand\expandafter<\expandafter>\csname#1\endcsname[#2][#3]{#4}%<- here
  \expandafter\long\expandafter\def\csname end#1\endcsname{#5}%
}

\renewcommand<>\beamer@columncom[2][\beamer@colmode]{%
  \beamer@colclose%
  \def\beamer@colclose{\end{minipage}\hfill\end{actionenv}\ignorespaces}%
  \begin{actionenv}#3%
  \setkeys{beamer@col}{#1}%
  \begin{minipage}[\beamer@colalign]{#2}%
  \leavevmode\raggedleft\beamer@colheadskip\ignorespaces}

\renewenvironment<>{columns}[1][]{%
  \begin{actionenv}#2%
    \def\beamer@colentrycode{%
      \hbox to\textwidth\bgroup%
      \leavevmode%
      \hskip-\beamer@leftmargin%
      \nobreak%
      \beamer@tempdim=\textwidth%
      \advance\beamer@tempdim by\beamer@leftmargin%
      \advance\beamer@tempdim by\beamer@rightmargin%
      \hbox to\beamer@tempdim\bgroup%
      \hbox{}\hfill\ignorespaces}%
    \def\beamer@colexitcode{\egroup%
      \nobreak%
      \hskip-\beamer@rightmargin\egroup}%
    \ifbeamer@centered\setkeys{beamer@col}{c}\else\setkeys{beamer@col}{t}\fi%
    \setkeys{beamer@col}{#1}%
    \par%
    \leavevmode\beamer@colentrycode%
    \def\beamer@colclose{}\ignorespaces}%
    {\beamer@colclose\def\beamer@colclose{}\beamer@colexitcode\end{actionenv}}%

%   {\def\thebibliography#1{\chapter*{\refname\@mkboth
%   {\uppercase{\refname}}{\uppercase{\refname}}}\list
%   {[\arabic{enumi}]}{\settowidth\labelwidth{[#1]}
%   \rightmargin\labelwidth
%   \advance\rightmargin\labelsep
%   \advance\rightmargin\bibindent
%   \itemindent -\bibindent

%   \listparindent \itemindent
%   \parsep \z@
%   \usecounter{enumi}}
%   \def\newblock{}
%   \sloppy
%   \sfcode`\.=1000\relax}}

%   \let\ORIG@BR@@lbibitem\BR@@lbibitem
% \apptocmd\ORIG@BR@@lbibitem{\endgroup}{}{}
% \def\BR@@lbibitem{\begingroup\catcode`\%=12 \ORIG@BR@@lbibitem}
% \makeatother
% %% برای سایر استیل‌ها
% \makeatletter
% \let\ORIG@BR@@bibitem\BR@@bibitem
% \apptocmd\ORIG@BR@@bibitem{\endgroup}{}{}
% \def\BR@@bibitem{\begingroup\catcode`\%=12 \ORIG@BR@@bibitem}


\makeatother

% hide navigation bar
\beamertemplatenavigationsymbolsempty

% fix allowframebreaks numbering
\setbeamertemplate{frametitle continuation}{ - \insertcontinuationcount}

\makeatletter
% reset footnot counter per new page
% \@newctr{footnote}[page]
\makeatother


% دستور های لازم برای تعریف ترجمهٔ دستورات الگوریتم
\makeatletter
\renewcommand{\algorithmicrequire}{\if@RTL\textbf{ورودی:}\else\textbf{Require:}\fi}
\renewcommand{\algorithmicensure}{\if@RTL\textbf{خروجی:}\else\textbf{Ensure:}\fi}
\renewcommand{\algorithmicend}{\if@RTL\textbf{پایان}\else\textbf{end}\fi}
\renewcommand{\algorithmicif}{\if@RTL\textbf{اگر}\else\textbf{if}\fi}
\renewcommand{\algorithmicthen}{\if@RTL\textbf{آنگاه}\else\textbf{then}\fi}
\renewcommand{\algorithmicelse}{\if@RTL\textbf{وگرنه}\else\textbf{else}\fi}
\renewcommand{\algorithmicfor}{\if@RTL\textbf{برای}\else\textbf{for}\fi}
\renewcommand{\algorithmicforall}{\if@RTL\textbf{برای هر}\else\textbf{for all}\fi}
\renewcommand{\algorithmicdo}{\if@RTL\textbf{انجام بده}\else\textbf{do}\fi}
\renewcommand{\algorithmicwhile}{\if@RTL\textbf{تا زمانی که}\else\textbf{while}\fi}
\renewcommand{\algorithmicloop}{\if@RTL\textbf{تکرار کن}\else\textbf{loop}\fi}
\renewcommand{\algorithmicrepeat}{\if@RTL\textbf{تکرار کن}\else\textbf{repeat}\fi}
\renewcommand{\algorithmicuntil}{\if@RTL\textbf{تا زمانی که}\else\textbf{until}\fi}
\renewcommand{\algorithmicprint}{\if@RTL\textbf{چاپ کن}\else\textbf{print}\fi}
\renewcommand{\algorithmicreturn}{\if@RTL\textbf{بازگردان}\else\textbf{return}\fi}
\renewcommand{\algorithmicand}{\if@RTL\textbf{و}\else\textbf{and}\fi}
\renewcommand{\algorithmicor}{\if@RTL\textbf{یا}\else\textbf{or}\fi}
\renewcommand{\algorithmicxor}{\if@RTL\textbf{یا (نه هردو)}\else\textbf{xor}\fi} % TODO add better translate
\renewcommand{\algorithmicnot}{\if@RTL\textbf{نقیض}\else\textbf{not}\fi}
\renewcommand{\algorithmicto}{\if@RTL\textbf{تا}\else\textbf{to}\fi}
\renewcommand{\algorithmicinputs}{\if@RTL\textbf{ورودی‌ها}\else\textbf{inputs}\fi}
\renewcommand{\algorithmicoutputs}{\if@RTL\textbf{خروجی‌ها}\else\textbf{outputs}\fi}
\renewcommand{\algorithmicglobals}{\if@RTL\textbf{متغیرهای عمومی}\else\textbf{globals}\fi}
\renewcommand{\algorithmicbody}{\if@RTL\textbf{انجام بده}\else\textbf{do}\fi}
\renewcommand{\algorithmictrue}{\if@RTL\textbf{درست}\else\textbf{true}\fi}
\renewcommand{\algorithmicfalse}{\if@RTL\textbf{نادرست}\else\textbf{false}\fi}
\renewcommand{\algorithmicendif}{\algorithmicend\textbf{ شرط }\algorithmicif}
\renewcommand{\algorithmicendfor}{\algorithmicend\textbf{ حلقهٔ }\algorithmicfor}
\renewcommand{\algorithmicendwhile}{\algorithmicend\textbf{ حلقهٔ }\algorithmicwhile}
\renewcommand{\algorithmicendloop}{\algorithmicend\textbf{ حلقهٔ }\algorithmicloop}
\renewcommand{\algorithmiccomment}[1]{\{{\itshape #1}\}}
\makeatletter

\makeatletter
% تعریف محیط بدون headline
\newenvironment{withoutheadline}{
  \setbeamertemplate{headline}[default]
  \def\beamer@entrycode{\vspace*{-\headheight}}
}{}
\makeatother

% برای درج عدد به جای شکل در beamer
\setbeamertemplate{bibliography item}{\insertbiblabel}

\renewcommand{\sectionname}{قسمت}
\renewcommand{\subsectionname}{زیرقسمت}
\def\subsubsectionname{\translate{Subsubsection}}
\def\insertsubsubsectionnumber{\arabic{subsubsection}}
%\def\subsubsectionpage{\usebeamertemplate*{subsubsection page}}
\AtBeginSection{\frame{\sectionpage}}
%\AtBeginSubsection{\frame{\subsectionpage}}
%\AtBeginSubsubsection{\frame{\subsubsectionpage}}


\def\Put(#1,#2)#3{\leavevmode\makebox(0,0){\put(#1,#2){#3}}}


% برای شفاف کردن مواردی که در
%\setbeamercovered{transparent}

\setbeamertemplate{frametitle}
{
	\nointerlineskip
	\begin{beamercolorbox}[sep=0.0cm,ht=1.6em,wd=\paperwidth]{frametitle}
		\vbox{}\vskip-1ex%
	\centering
%	\raisebox{-1em}{\includegraphics[height=1.6em]{./img/eng-logo.png}} \hfill
		{  \strut\insertframetitle\strut}
%		\hfill		\includegraphics[height=1.6em]{./img/logo.png}
		\vskip+0.2ex %		\vskip-0.8ex%
	\end{beamercolorbox}
}


\makeatletter
\setbeamertemplate{footline}
{
	\leavevmode%
	\hbox{\tiny%\fontsize{7}{8}\selectfont%
		\begin{beamercolorbox}[wd=.3\paperwidth,ht=2.25ex,dp=1ex,center]{author in head/foot}
			\hspace*{1ex}\usebeamerfont{author in head/foot}\insertshortauthor~~\beamer@ifempty{\insertshortinstitute}{}{(\insertshortinstitute)}\hspace*{1ex}
		\end{beamercolorbox}%
		\begin{beamercolorbox}[wd=.5\paperwidth,ht=2.25ex,dp=1ex,center]{title in head/foot}
			\usebeamerfont{title in head/foot}\insertshorttitle
		\end{beamercolorbox}%
		\begin{beamercolorbox}[wd=.2\paperwidth,ht=2.25ex,dp=1ex,right]{date in head/foot}
			\hfill
			\usebeamerfont{date in head/foot}\insertshortdate{}
			\hfill % \hspace*{4ex}% original: 2ex
			\inserttotalframenumber / \insertframenumber{} \hspace*{1ex}% original: 2ex
	\end{beamercolorbox}}%
	\vskip0pt%
}
\makeatother
\makeatletter
\setbeamerfont{frametitle}{size=\small}
\makeatother
%% simple headline for one row sectino name and one row subsection name
%\setbeamertemplate{headline}{
%	\begin{beamercolorbox}[wd=\paperwidth,ht=2.5ex,dp=1.125ex]{section in head/foot}%
%		\hspace{3ex}{\insertsectionhead}
%	\end{beamercolorbox}
%	\begin{beamercolorbox}[ht=2.5ex,dp=1.125ex,leftskip=.3cm,rightskip=.3cm plus1fil]{subsection in head/foot}
%		\usebeamerfont{subsection in head/foot}\insertsubsectionhead
%\end{beamercolorbox}}

%% add section navigation above frame title (miniframe style)
\setbeamertemplate{headline}{%
    \begin{beamercolorbox}{section in head/foot}
        \vskip0pt\insertnavigation{\paperwidth}\vskip2pt
    \end{beamercolorbox}%
}
% add section navigation above frame title
%\setbeamertemplate{headline}
%{
%	\begin{beamercolorbox}{section in head/foot}
%		\vskip2pt\insertsectionnavigationhorizontal{\textwidth}{}{}\vskip2pt
%	\end{beamercolorbox}
%}
\setbeamertemplate{section in head/foot}{\color{fg}\insertsectionhead}
\setbeamertemplate{section in head/foot shaded}{\color{fg!50!bg}\insertsectionhead}
\setbeamercolor{section in head/foot}{fg=white}

%%% can add logo to section frame
%% \newcommand{\secimage}{example-image-a}
%%% can change in document with
%% \renewcommand{\secimage}{example-image-b}
%\AtBeginSection[]{
%	\begin{frame}
%		\vfill
%		\centering
%%       \includegraphics[width=4cm]{\secimage}
%		\begin{beamercolorbox}[sep=8pt,center,shadow=true,rounded=true]{title}
%			\usebeamerfont{title}\insertsectionhead\par%
%		\end{beamercolorbox}
%		\vfill
%	\end{frame}
%}


\title
 [شکستن پروتکل‌های رمزنگاری با استفاده از حملات کوانتومی متغیر پیشرفته]
{شکستن پروتکل‌های رمزنگاری با استفاده از حملات کوانتومی متغیر پیشرفته}

\author[محمدعلی خواجه‌ئیان]{محمدعلی خواجه‌ئیان}
\subtitle{دفاع پروپوزال}
\institute[دانشکده فنی/دانشگاه تهران]{
    استاد راهنما: زهرا شاطرزاده‌یزدی \\
    دانشکدهٔ علوم مهندسی  /  دانشگاه تهران 
}
%\date{زمستان ۱۳۹۹}
\subject{مهندسی کامپیوتر}

% \AtBeginDocument{
    \makeatletter
    \hypersetup{
        pdftitle={شکستن پروتکل‌های رمزنگاری با استفاده از حملات کوانتومی متغیر پیشرفته},
        pdfauthor={محمدعلی خواجه‌ئیان},
        pdfsubject={Thesis in Computer Science},
%        pdfkeywords={\@latinkeywords},
        pdfdirection={R2L}
    }
    \makeatother
% }

\titlegraphic{
    \vspace{-2cm}
    \makebox[0.9\paperwidth]{
        \includegraphics[height=1.4cm]{./img/eng-logo.png}
        \hfill
        \includegraphics[height=1.4cm]{./img/logo.png}
        \vspace{5 mm}
    }
}


\begin{document}

\begin{withoutheadline}
\frame{\maketitle}
\end{withoutheadline}


 \begin{frame}
 	\frametitle{فهرست}
     % \raggedright
 	\tableofcontents
 \end{frame}

%%============multicolumns TOC ==============
%\setbeamertemplate{section in toc}[circle]
\begin{frame}{فهرست}%to allow TOC break
    \begin{columns}[onlytextwidth]
        \begin{column}{.1\textwidth}\end{column}
        \begin{column}{.3\textwidth}
            % \raggedleft
            \tableofcontents[sections=1-3, subsubsectionstyle=hide/hide,hideothersubsections]
        \end{column}
        \begin{column}{.3\textwidth}
            % \raggedleft
            \tableofcontents[sections=4-6, subsubsectionstyle=hide/hide,hideothersubsections]%hideothersubsections
        \end{column}
        \begin{column}{.3\textwidth}
            % \raggedleft
            \tableofcontents[sections=7-9, subsubsectionstyle=hide/hide,hideothersubsections]%hideothersubsections
        \end{column}
    \end{columns}
\end{frame}
%============End multicolumns TOC ==============

\section{تعریف مسئله}
\begin{frame}
    \frametitle{مقدمه}

    \begin{columns}[onlytextwidth]
        \setbeamercolor{alerted text}{fg=blue}
        \setbeamerfont{alerted text}{series=\bfseries}
        \setbeamercovered{transparent}
        \begin{column}{.4\textwidth}
            % \raggedleft
            \begin{moredi}
                \item نمونه از یک لیست دولایه در کنار یک تصویر
                \begin{moredi}[<1>]
                    \item در این لیست موارد زیادی می‌تواند قرار بگیرد
                    \item مثلاً
                    \item $\cdots$

                \end{moredi}
            \end{moredi}
        \end{column}
        \begin{column}{.6\textwidth}
            % \raggedleft
            \begin{moredi}
                \item این مورد برای یک ترکیب دولایه ای آماده شده
                \begin{moredi}
                    \item<1> این مورد فقط در قسمت اول دیده می‌شود
                    \item \alert<2>{این مورد تأکیدی در صفحهٔ دوم دیده می‌شود}
                    \item<1> موارد بیشتر

                \end{moredi}
            \end{moredi}
            \begin{figure}
                % \includegraphics[width=\textwidth]{./img/social-sais.jpg}
                \vspace{-1em}
                \includegraphics[height=0.35\textheight]{logo.png}
                \caption{اولین تصویر }
            \end{figure}
        \end{column}
        % \begin{column}{.01\textwidth}\end{column}
    \end{columns}

\end{frame}

\section{اهداف}
\subsection{زنجیرهٔ مارکوف}
\begin{frame}
    \frametitle{زنجیرهٔ مارکوف}

    زنجیرهٔ مارکوف\LTRfootnote{Markov Chain}
    \begin{moredi}
        \item مدلی برای توصیف توالی رخدادهای احتمالی (\textbf{فرایند تصادفی}\LTRfootnote{Stochastic process})
        \item احتمال هر رخداد فقط به وضعیت رخداد قبلی خود وابسته (\textbf{بدون حافظه}\LTRfootnote{Memory less})
        \item قابل تعریف در دو حالت: زمان گسسته و زمان پیوسته
    \end{moredi}
    % \begin{eqnarray}
    %     \label{eqn:discreate-time-markov-chain}
    %     \Pr(X_{n+1}=x\mid X_{1}=x_{1},X_{2}=x_{2},\ldots ,X_{n}=x_{n})=\Pr(X_{n+1}=x\mid X_{n}=x_{n}) \\
    %     \label{eqn:continous-time-markov-chain}
    %     \Pr(X_{t_{n+1}}=i_{n+1}\mid X_{t_{0}}=i_{0},X_{t_{1}}=i_{1},\ldots ,X_{t_{n}}=i_{n})=p_{i_{n}i_{n+1}}(t_{n+1}-t_{n})
    % \end{eqnarray}
    \pause
    \begin{table}
        \caption{حالت‌های معروف برای مدل مارکوف}
        \vspace{-1em}
        \small
        \begin{tabular}{|c|c|c|}
            \hline
            \textbf{حالت‌ها}       & \textbf{زمان پیوسته}       & \textbf{زمان گسسته}        \\
            \hline
            \textbf{وضعیت گسسته}  & فرایند مارکوف              & زنجیره مارکوف              \\
            \hline
            \textbf{وضعیت پیوسته} & فرایند مارکوف وضعیت پیوسته & زنجیره مارکوف وضعیت پیوسته \\
            \hline
        \end{tabular}
    \end{table}

\end{frame}

\subsection{نظریهٔ میدان متوسط}
\begin{frame}
    \frametitle{نظریهٔ میدان متوسط (\lr{MFT}\LTRfootnote{Mean Field Theory})}

    \begin{moredi}
        \item رفتار مدل‌های بزرگ و پیچیدهٔ تصادفی را به کمک یک مدل ساده‌تر
        \begin{moredi}
            \item تبدیل یک مسئله با تعداد بسیار زیادی از اجزای کوچک که با یکدیگر در ارتباط هستند و رفتار تصادفی دارند
            \item به یک مسئله سادهٔ تک ذره‌ای
            \item تحلیل رفتار میانگین کل ذرات را مدل می‌کند
        \end{moredi}
        \item تبدیل و تحلیل یک مسئلهٔ بین ذره‌ای برای تعداد بی‌شمار ذره به یک روش تک ذره‌ای
    \end{moredi}
    \begin{figure}
        \begin{tikzpicture}
            {\fill (3.1,0) circle (0.1);}
            \draw[thick,->] (1,0) -- node[above] {MFT} (3,0) ;
            \draw (0,0) ellipse (1 and 1);
            \clip (0,0) ellipse (1 and 1);
            \pgfmathsetseed{24122015}
            \foreach \p in {1,...,500} { \fill (1*rand,1*rand) circle (0.02);  }
        \end{tikzpicture}
        \caption{تبدیل مسئله بسیار ذره‌ای به تک ذره‌ای برای تحلیل رفتار کل ذرات در کنار هم به کمک نظریهٔ میدان متوسط}
        \label{fig:mft-sample}
    \end{figure}

\end{frame}

\subsection{مدل‌های انتشار بیماری}
\begin{frame}
    \frametitle{مدل اولیهٔ مستعد-بیمار-ایمن (\lr{SIR})}

    \begin{columns}[onlytextwidth]
        \begin{column}{.5\textwidth}
            \begin{moredi}
                \item مدل SIR در سال ۱۹۲۷ میلادی، توسط آقای کِرماک\footnote[frame]{W. O. Kermack} و آقای مَک‌کِندریک\footnote[frame]{A. G. McKendrick}
                \begin{moredi}
                    \item سالم (در معرض ابتلا) در قالب $S(t)$
                    \item مبتلا در قالب $I(t)$
                    \item بهبود یافته (یا ایمن) در قالب $R(t)$
                \end{moredi}
            \end{moredi}
            \vspace{-1em}

        \end{column}
        \begin{column}{.5\textwidth}
            \begin{figure}
                \centering
                \begin{tikzpicture}[thick,scale=1.1, every node/.style={scale=1}]
                    \node[state] (s) at (-2, 0 ) {S};
                    \node[state] (i) at ( 0 ,0 ) {I};
                    \node[state] (r) at ( 2, 0 ) {R};

                    \draw[->] (s) edge node [anchor=north] {$\beta$} (i);
                    \draw[->] (i) edge[above] node [anchor=north] {$\gamma$} (r);
                \end{tikzpicture}
                \caption{مدل مارکوف انتشار بیماری SIR}
                \label{fig:SIR}
            \end{figure}
            \begin{equation}
                \label{eqn:sir-ode-formulation}
                \begin{aligned}
                    {\frac {dS}{dt}} & =-{\frac {\beta SI}{N}}         \\
                    \frac {dI}{dt}   & ={\frac {\beta SI}{N}}-\gamma I \\
                    \frac {dR}{dt}   & =\gamma I
                    %                    N                & =S+I+R
                \end{aligned}
            \end{equation}
        \end{column}
    \end{columns}

\end{frame}

\begin{frame}
    \frametitle{مدل‌های معروف دیگر}

    \begin{moredi}
        \setbeamercovered{transparent}
        \item<1> \lr{SIS}: بازگشت به حالت مستعد پس از بیماری
        \item<1> \lr{SIRS}:‌ بازگشت به دورهٔ مستعد پس از یک دورهٔ مشخص
        \item<1-2> \lr{SEIS}: وجود یک دورهٔ نهان و بدون علامت پس از ابتلا و قبل از بروز عفونت
        \item<1> \lr{MSIR}: در نظر گرفتن وضعیت مصونیت کودکان در مقابل بیماری
        \item<1-2> \lr{SAIS}: در نظر گرفتن وضعیت آگاه برای کاهش نرخ ابتلا
        \item<1> \lr{SIRC}: با وضعیت ناقل\LTRfootnote{Carrier}
        \item<1> \lr{SIRV}: با وضعیت هوشیاری\LTRfootnote{Vigilant}

    \end{moredi}

\end{frame}

\begin{frame}
    \frametitle{مدل مستعد-آگاه-بیمار-مستعد (\lr{SAIS})}

    \begin{columns}[onlytextwidth]
        \begin{column}{.5\textwidth}
            \begin{moredi}
                % \item یک وضعیت جدید برای هر گره در نظر گرفته می‌شود که فرد با آگاهی و آمادگی از خطر بیماری و رعایت نکات پیشگیری
                \item کاهش نرخ ابتلا از $\beta$ به $\beta_a$ برای افراد آگاه و مراقب
                \item تغییر وضعیت به حالت آگاه و مراقب با نرخ $\kappa$
            \end{moredi}
            \begin{figure}[ht]
                \centering
                \includegraphics[width=\textwidth]{./img/sais-transition.pdf}
                \caption{تغییرات گذرا برای مدل SAIS}
                \label{fig:sais-transition}
            \end{figure}
            % \vspace{-1em}
            % \begin{eqnarray}
            %     \label{eq:sais-math}
            %     \dfrac{dS}{dt}&=&\gamma I - (\beta+\kappa) S\dfrac{I}{N}\\
            %     \dfrac{dA}{dt}&=&\kappa S\dfrac{I}{N} - \beta_a A\dfrac{I}{N}\\
            %     \dfrac{dI}{dt}&=&(\beta S + \beta_a A)\dfrac{I}{N} - \gamma I\\
            %     \label{eq:sais-math-end}
            %     N&=&S+A+I
            % \end{eqnarray}
        \end{column}
        \begin{column}{.5\textwidth}
            \begin{figure}[ht]
                \centering
                \begin{tikzpicture}[thick,scale=1.1, every node/.style={scale=1.0}]
                    \node[state] (s) at (-1.5, 0 ) {S};
                    \node[state] (a) at ( 0 ,-2 ) {A};
                    \node[state] (i) at ( 1.5, 0 ) {I};

                    \draw[->] (i) edge[bend left, below] node [anchor=south] {$\gamma$} (s);
                    \draw[->] (s) edge[bend left, above] node [anchor=north] {$\beta$} (i);
                    \draw[->] (a) edge[bend right, above] node [anchor=east] {$\beta_a$} (i);
                    \draw[->] (s) edge[bend right, above] node [anchor=west] {$\kappa$} (a);
                \end{tikzpicture}
                \caption{مدل مارکوف انتشار بیماری SAIS}
                \label{fig:SAIS}
            \end{figure}
        \end{column}
    \end{columns}

\end{frame}

\subsection{منطق فازی}
\begin{frame}[allowframebreaks]
    \frametitle{منطق فازی}

    \begin{moredi}
        \item روشی برای مدل کردن ارتباط بین ورودی و خروجی
        \item تعریف \textbf{مجموعهٔ فازی}\LTRfootnote{Fuzzy Set}
        \item اعضای مجموعهٔ فازی شامل \textbf{متغیرهای زبانی} هستند که مقادیر آن‌ها از \textbf{مقادیر زبانی}\LTRfootnote{Linguistic values} انتخاب می‌شود.
        \item تعریف مقدار حدودی بین ۰ تا ۱ برای ورودی و خروجی‌ها (\textbf{درجهٔ عضویت}\LTRfootnote{Membership grade})
    \end{moredi}

    \begin{figure}[ht]
        \centering
        \includegraphics[height=.3\textheight]{Fuzzy_logic_temperature_en.pdf}
        \caption{تابع عضویت فازی برای دمای محیط}
        \label{fig:fuzzy-logic-temperature}
    \end{figure}

\end{frame}

\begin{frame}
    \frametitle{روش عملکرد منطق فازی}

    \begin{enumerate}\RTList
        \item تبدیل ورودی‌های عددی به \textbf{متغیرهای زبانی}\LTRfootnote{Linguistic variable} (غیر دقیق و حسی) یا \textbf{فازی سازی}
              \begin{moredi}
                  \item بر اساس  \textbf{تابع عضویت فازی}
              \end{moredi}
        \item \textbf{استنتاج فازی} مطابق با \textbf{قواعد فازی} تعریف شده (بر اساس توصیف زبانی اگر $\leftarrow$ آنگاه)
        \item تبدیل خروجی فازی به یک متغیر عددی (\textbf{فازی گشایی})
              \begin{moredi}
                  \item بر اساس  \textbf{تابع عضویت فازی}
                  \item به کمک روش‌های تجمیع سازی نتایج
              \end{moredi}
    \end{enumerate}

\end{frame}

\section{ضرورت انجام پژوهش}
\begin{frame}
    \frametitle{کارهای پیشین}

    دسته بندی کارهای پیشین در زمینهٔ شبکه‌های اجتماعی و انتشار بیماری  یا ویروس:
    {
    \setbeamercovered{transparent}
    \begin{enumerate}\RTList
        \item<1-2> انتشار بیماری
        \item ساختار عمومی انتشار بیماری
        \item تأثیر گذاری اجتماعی و نفوذ فکری
        \item تغییرات آگاهی و رفتار اجتماعی
        \item<1> گراف پویا و تغییرات یال و گره
        \item<1> تعادل و پایداری گراف
        \item<1> کنترل شبکه و تغییر سیاست
        \item<1> پیش‌بینی انتشار بیماری
        \item<1> انتشار ویروس و بد افزارِ رایانه‌ای

    \end{enumerate}
    }
\end{frame}

\begin{frame}
    \frametitle{انتشار بیماری}

    \begin{moredi}
        \item بررسی مدل آشکار و نهان بر میزان شیوع جامعه \cite{chen2020time}
        \item بررسی مدل \lr{SEIR} برای بیماری کووید-۱۹ با توجه به ارتباط‌های بین شهری و بین کشوری در اروپا \cite{wang2019coevolution,estrada2020covid}
        \item بررسی نویز (خطا در اطلاعات ورودی) و تأثیر آن بر نتیجهٔ تحلیل مدل \lr{SIS} \cite{vizuete2020graphon}
        \item در نظر گرفتن واکسیناسیون در مدل \lr{SIS} \cite{khanjanianpak2020competition}
        \item تطبیق اطلاعات بیماری کووید-۱۹ در کشور فرانسه بر روی مدل \lr{SEIR} \cite{efimov2020interval}
    \end{moredi}

\end{frame}

\begin{frame}
    \frametitle{ساختار عمومی انتشار بیماری}

    \begin{moredi}
        \item ساختار عمومی انتشار بیماری برای مدل‌های رایج (مثل \lr{SIS, SAIS}) \cite{moon2020group}
        \item بررسی ساختارهای متداول بیماری بر روی شبکه‌های چند لایه \cite{abhishek2020sis}
    \end{moredi}

\end{frame}

\begin{frame}
    \frametitle{تأثیر گذاری اجتماعی و نفوذ فکری}

    \begin{moredi}
        \item تحلیل انتشار شایعه در شبکه‌های اجتماعی برخط با در نظر گرفتن مدل نظریهٔ بازی \cite{huang2020developing}
        \item ارائهٔ یک مدل شبیه‌سازی برای بررسی شرایط و نتیجه رسیدن به اجماع  در یک شبکهٔ برخط با دو گروه فکری مخالف با در نظر گرفتن کیفیت ارتباط‌ها \cite{bolzern2020opinion,nettasinghe2019diffusion}
        \item بررسی تأثیر اخبار انتشار بیماری کووید-۱۹ در شبکه‌های اجتماعی برخط \cite{cinelli2020covid}
    \end{moredi}

\end{frame}

\begin{frame}
    \frametitle{تغییرات آگاهی و رفتار اجتماعی}

    \begin{moredi}
        \item بررسی مدل بیماری \lr{SEIV}\LTRfootnote{Susceptible-Exposed-Infected-Vigilant} برای یک شبکه و تأثیر هوشیاری افراد بر تعداد ارتباط‌های فعال با دیگران و زمان رسیدن به حالت پایدار بدون بیماری \cite{li2019control,bhowmick2020influence}
        \item تأثیر آگاهی و میزان شیوع بیماری در ارتباط بین افراد در یک شبکهٔ دو لایه (یک لایه ثابت و یک لایهٔ متغیر)\cite{sahneh2019contact}
    \end{moredi}

\end{frame}

\section{پرسش های پژوهش}
\begin{frame}
    \frametitle{فرایند کلی حل مسئله}
    {
        \setbeamercovered{transparent}
        شبیه‌سازی:
        \begin{enumerate}\RTList
            \item<-1> تصادفی (محاسبهٔ وضعیت و شرایط جدید هر گره و به روز کردن همه گره‌ها در یک لحظه)
            \item<2-> آماری (محاسبهٔ امید ریاضی و میانگین وضعیت و شرایط انتقال برای کل شبکه در مدل مارکوف)

        \end{enumerate}

        مدل‌سازی:
        \begin{enumerate}\RTList
            \item تعریف متغیرهای فازی و توابع عضویت (فضای پیوسته)
            \item تعریف جدول قواعد فازی (ارتباط بین ورودی و خروجی‌های مسئله)
            \item تعریف روابط ریاضی تجمیع سازی برای هر گره
            \item<2-> تعریف مدل مارکوف معادل
            \item<2-> تعریف روابط آماری و کلی (مبتنی بر نظریهٔ میدان متوسط)
            \item<1> تعریف الگوی بیماری
            \item شبیه‌سازی

        \end{enumerate}
    }
\end{frame}

\section{روش و فنون پژوهش}
\subsection{بخش تئوری}
\subsection{بخش پیاده سازی}

\section{زمانبدی پیشنهادی }
\section{پیشینه پژوهش}

\section{منابع و مراجع}
\begin{frame}[allowframebreaks,noframenumbering]
    %    \frametitle{منابع}
    %    \nocite{*} % to add all reference in bib
    % \printbibliography
    \begin{latin}
        \bibliographystyle{unsrt-fa} % ieeetr-plain=unsrt
        \scriptsize
        \bibliography{./references}
    \end{latin}

\end{frame}

% \appendix
\section{صفحات پشتیبان}

\begin{frame}

    \begin{example}
    این یک مثال است.
    \end{example}

    \begin{definition}
    این یک تعریف است.
    \end{definition}

    \begin{theorem}
    این یک قضیه است.
    \end{theorem}
\end{frame}

\begin{frame}
\frametitle{اثبات ریاضی}
\begin{theorem}<1->[Pythagoras]
$ a^2 + b^2 = c^2$
یا
    \[c^2=a^2+b^2\]
\end{theorem}
\begin{proof}<2->
$\omega +\phi = \epsilon $
\end{proof}
\begin{corollary}<3->
$ x + y = y + x  $
\end{corollary}
\end{frame}

\begin{frame}[noframenumbering]
    \frametitle{نتیاج شبیه‌سازی آماری}

    \begin{figure}[htb]
        %	\vspace{-2cm}
        \centering
        \subfloat[یادگیری=$0.75$ و فراموشی=$0.5$]{\includegraphics[width=0.48\textwidth]{ode-75-50.pdf}}\
        \subfloat[یادگیری=$0.5$ و فراموشی $0.75$]{\includegraphics[width=0.48\textwidth]{ode-50-75.pdf}}
        \caption{نتیجهٔ اجرای شبیه‌سازی آماری در دو حالت}
        \label{fig:ode-sais}
    \end{figure}

\end{frame}

\begin{frame}[noframenumbering]
    \frametitle{}

    \small
    \begin{algorithm}[H]
        \caption{الگوریتم اجرای برنامهٔ شبیه‌سازی برای حالت امید ریاضی}
        \label{alg:simulation-expectation}
        \begin{algorithmic}[1]
            \REQUIRE زمان $t_{max}$ به عنوان زمان لازم برای انجام شبیه‌سازی،\\
            \REQUIRE توزیع درجهٔ گراف برای شبیه‌سازی،
            \ENSURE ماتریس تغییرات گراف از لحظهٔ ۰ تا $t_{max}$.
            \FOR {t از ۰ \TO $t_{max}$}
            \STATE محاسبهٔ نرخ انتقال بیماری
            \STATE محاسبهٔ نرخ یادگیری-فراموشی
            \STATE محاسبهٔ وضعیت جدید مدل مارکوف بیماری و آگاهی
            \ENDFOR
            \RETURN {ماتریس تغییرات زمانی}
        \end{algorithmic}
    \end{algorithm}

\end{frame}

\end{document}



