% !TeX spellcheck = fa_FA
% !TEX TS-program = xelatex
% !TEX encoding = UTF-8 Unicode

\documentclass[xcolor=dvipsnames, professionalfonts, aspectratio=169, 11pt]{beamer}

\usepackage{multicol}
\usepackage{tikz}
\usetikzlibrary{automata, positioning, arrows}
\usepackage{dblfnote}
\usepackage{float}
\usepackage{setspace}

%\hypersetup{colorlinks = true}

% ---------------------------------------
% color
% ---------------------------------------

\usefonttheme{serif}
\usetheme[compress]{Ilmenau} 
% \usetheme{Madrid}
%% hide subsection row in headline in Ilmenau
\makeatletter
 \beamer@theme@subsectionfalse
\makeatother
% \usetheme{AnnArbor}
\usetheme{Frankfurt}
% \usetheme{CambridgeUs}

% \usecolortheme{wolverine}

% cool and blue
% \usecolortheme{seahorse}

% \usecolortheme{rose}

%% Warm and red theme
% \usecolortheme{beaver}
\setbeamercolor{section number projected}{bg=blue,fg=white}
\setbeamercolor*{item}{fg=blue}

%% green theme
% \usecolortheme{spruce}
% \setbeamercolor{section number projected}{bg=OliveGreen,fg=white}
% \setbeamercolor*{item}{fg=OliveGreen}

% \usecolortheme[named=green]{structure}
%\setbeamercolor{normal text}{fg=blue}\usebeamercolor*{normal text}

\include{eslahBeamer}

% برای شفاف کردن مواردی که در
%\setbeamercovered{transparent}

\setbeamertemplate{frametitle}
{
	\nointerlineskip
	\begin{beamercolorbox}[sep=0.0cm,ht=1.6em,wd=\paperwidth]{frametitle}
		\vbox{}\vskip-1ex%
	\centering
%	\raisebox{-1em}{\includegraphics[height=1.6em]{./img/eng-logo.png}} \hfill
		{  \strut\insertframetitle\strut}
%		\hfill		\includegraphics[height=1.6em]{./img/logo.png}
		\vskip+0.2ex %		\vskip-0.8ex%
	\end{beamercolorbox}
}


\makeatletter
\setbeamertemplate{footline}
{
	\leavevmode%
	\hbox{\tiny%\fontsize{7}{8}\selectfont%
		\begin{beamercolorbox}[wd=.3\paperwidth,ht=2.25ex,dp=1ex,center]{author in head/foot}
			\hspace*{1ex}\usebeamerfont{author in head/foot}\insertshortauthor~~\beamer@ifempty{\insertshortinstitute}{}{(\insertshortinstitute)}\hspace*{1ex}
		\end{beamercolorbox}%
		\begin{beamercolorbox}[wd=.5\paperwidth,ht=2.25ex,dp=1ex,center]{title in head/foot}
			\usebeamerfont{title in head/foot}\insertshorttitle
		\end{beamercolorbox}%
		\begin{beamercolorbox}[wd=.2\paperwidth,ht=2.25ex,dp=1ex,right]{date in head/foot}
			\hfill
			\usebeamerfont{date in head/foot}\insertshortdate{}
			\hfill % \hspace*{4ex}% original: 2ex
			\inserttotalframenumber / \insertframenumber{} \hspace*{1ex}% original: 2ex
	\end{beamercolorbox}}%
	\vskip0pt%
}
\makeatother
\makeatletter
\setbeamerfont{frametitle}{size=\small}
\makeatother
%% simple headline for one row sectino name and one row subsection name
%\setbeamertemplate{headline}{
%	\begin{beamercolorbox}[wd=\paperwidth,ht=2.5ex,dp=1.125ex]{section in head/foot}%
%		\hspace{3ex}{\insertsectionhead}
%	\end{beamercolorbox}
%	\begin{beamercolorbox}[ht=2.5ex,dp=1.125ex,leftskip=.3cm,rightskip=.3cm plus1fil]{subsection in head/foot}
%		\usebeamerfont{subsection in head/foot}\insertsubsectionhead
%\end{beamercolorbox}}

%% add section navigation above frame title (miniframe style)
\setbeamertemplate{headline}{%
    \begin{beamercolorbox}{section in head/foot}
        \vskip0pt\insertnavigation{\paperwidth}\vskip2pt
    \end{beamercolorbox}%
}
% add section navigation above frame title
%\setbeamertemplate{headline}
%{
%	\begin{beamercolorbox}{section in head/foot}
%		\vskip2pt\insertsectionnavigationhorizontal{\textwidth}{}{}\vskip2pt
%	\end{beamercolorbox}
%}
\setbeamertemplate{section in head/foot}{\color{fg}\insertsectionhead}
\setbeamertemplate{section in head/foot shaded}{\color{fg!50!bg}\insertsectionhead}
\setbeamercolor{section in head/foot}{fg=white}

%%% can add logo to section frame
%% \newcommand{\secimage}{example-image-a}
%%% can change in document with
%% \renewcommand{\secimage}{example-image-b}
%\AtBeginSection[]{
%	\begin{frame}
%		\vfill
%		\centering
%%       \includegraphics[width=4cm]{\secimage}
%		\begin{beamercolorbox}[sep=8pt,center,shadow=true,rounded=true]{title}
%			\usebeamerfont{title}\insertsectionhead\par%
%		\end{beamercolorbox}
%		\vfill
%	\end{frame}
%}


\title
 [شکستن پروتکل‌های رمزنگاری با استفاده از حملات کوانتومی سازگارپذیر پیشرفته]
{شکستن پروتکل‌های رمزنگاری با استفاده از حملات کوانتومی سازگارپذیر پیشرفته}

\author[محمدعلی خواجه‌ئیان]{محمدعلی خواجه‌ئیان}
\subtitle{}
\institute[دانشکده فنی/دانشگاه تهران]{
    استاد راهنما: زهرا شاطرزاده‌یزدی \\
    دانشکدهٔ علوم مهندسی  /  دانشگاه تهران 
}
%\date{زمستان ۱۳۹۹}
\subject{مهندسی کامپیوتر}

% \AtBeginDocument{
    \makeatletter
    \hypersetup{
        pdftitle={شکستن پروتکل‌های رمزنگاری با استفاده از حملات کوانتومی سازگارپذیر پیشرفته},
        pdfauthor={محمدعلی خواجه‌ئیان},
        pdfsubject={Thesis in Computer Science},
%        pdfkeywords={\@latinkeywords},
        pdfdirection={R2L}
    }
    \makeatother
% }

\titlegraphic{
    \vspace{-2cm}
    \makebox[0.9\paperwidth]{
        \includegraphics[height=1.4cm]{./img/eng-logo.png}
        \hfill
        \includegraphics[height=1.4cm]{./img/logo.png}
        \vspace{5 mm}
    }
}


\begin{document}

\begin{withoutheadline}
\frame{\maketitle}
\end{withoutheadline}


 \begin{frame}
 	\frametitle{فهرست}
     % \raggedright
 	\tableofcontents
 \end{frame}

%%============multicolumns TOC ==============
%\setbeamertemplate{section in toc}[circle]
\begin{frame}{فهرست}%to allow TOC break
    \begin{columns}[onlytextwidth]
        \begin{column}{.1\textwidth}\end{column}
        \begin{column}{.3\textwidth}
            % \raggedleft
            \tableofcontents[sections=1-3, subsubsectionstyle=hide/hide,hideothersubsections]
        \end{column}
        \begin{column}{.3\textwidth}
            % \raggedleft
            \tableofcontents[sections=4-6, subsubsectionstyle=hide/hide,hideothersubsections]%hideothersubsections
        \end{column}
        \begin{column}{.3\textwidth}
            % \raggedleft
            \tableofcontents[sections=7-9, subsubsectionstyle=hide/hide,hideothersubsections]%hideothersubsections
        \end{column}
    \end{columns}
\end{frame}
%============End multicolumns TOC ==============

\section{تعریف مسئله}
\begin{frame}
    \frametitle{تعریف مسئله}

    \textbf{الگوریتم های کوانتومی سازگارپذیر\LTRfootnote{Variational Quantum Algorithms}} که از ترکیب پردازش کلاسیک و کوانتومی استفاده می‌کنند، گزینه‌ای مناسب برای \textbf{رایانه‌های کوانتومی اندازه‌میانی پراختلال}\LTRfootnote{Noisy Intermediate Scale Quantum Device} هستند. این پژوهش کارایی الگوریتم های کوانتومی سازگارپذیر را در شکستن \textbf{رمزنگاری یکسان‌کلید}\LTRfootnote{Symmetric-Key Cryptography} و بهینه‌سازی این حمله‌ها بررسی می‌کند.

\end{frame}

\section{اهداف}
\begin{frame}
    \frametitle{اهداف پژوهش}
    اهداف اصلی این پروژه به شرح زیر هستند:


\end{frame}

\begin{frame}
    \frametitle{اهداف پژوهش}
    \begin{table}
        \caption{حالت‌های معروف برای مدل مارکوف}
        \vspace{-1em}
        \small
        \begin{tabular}{|c|c|c|}
            \hline
            \textbf{حالت‌ها}       & \textbf{زمان پیوسته}       & \textbf{زمان گسسته}        \\
            \hline
            \textbf{وضعیت گسسته}  & فرایند مارکوف              & زنجیره مارکوف              \\
            \hline
            \textbf{وضعیت پیوسته} & فرایند مارکوف وضعیت پیوسته & زنجیره مارکوف وضعیت پیوسته \\
            \hline
        \end{tabular}
    \end{table}

\end{frame}


\section{ضرورت انجام پژوهش}
\begin{frame}
    \frametitle{ضرورت انجام پژوهش}

    دسته بندی کارهای پیشین در زمینهٔ شبکه‌های اجتماعی و انتشار بیماری  یا ویروس:
    {
    \setbeamercovered{transparent}
    \begin{enumerate}\RTList
        \item<1-2> انتشار بیماری
        \item ساختار عمومی انتشار بیماری
        \item تأثیر گذاری اجتماعی و نفوذ فکری
        \item تغییرات آگاهی و رفتار اجتماعی
        \item<1> گراف پویا و تغییرات یال و گره
        \item<1> تعادل و پایداری گراف
        \item<1> کنترل شبکه و تغییر سیاست
        \item<1> پیش‌بینی انتشار بیماری
        \item<1> انتشار ویروس و بد افزارِ رایانه‌ای

    \end{enumerate}
    }
\end{frame}

\begin{frame}
    \frametitle{انتشار بیماری}

    \begin{moredi}
        \item بررسی مدل آشکار و نهان بر میزان شیوع جامعه \cite{chen2020time}
        \item بررسی مدل \lr{SEIR} برای بیماری کووید-۱۹ با توجه به ارتباط‌های بین شهری و بین کشوری در اروپا \cite{wang2019coevolution,estrada2020covid}
        \item بررسی نویز (خطا در اطلاعات ورودی) و تأثیر آن بر نتیجهٔ تحلیل مدل \lr{SIS} \cite{vizuete2020graphon}
        \item در نظر گرفتن واکسیناسیون در مدل \lr{SIS} \cite{khanjanianpak2020competition}
        \item تطبیق اطلاعات بیماری کووید-۱۹ در کشور فرانسه بر روی مدل \lr{SEIR} \cite{efimov2020interval}
    \end{moredi}

\end{frame}

\begin{frame}
    \frametitle{ساختار عمومی انتشار بیماری}

    \begin{moredi}
        \item ساختار عمومی انتشار بیماری برای مدل‌های رایج (مثل \lr{SIS, SAIS}) \cite{moon2020group}
        \item بررسی ساختارهای متداول بیماری بر روی شبکه‌های چند لایه \cite{abhishek2020sis}
    \end{moredi}

\end{frame}

\begin{frame}
    \frametitle{تأثیر گذاری اجتماعی و نفوذ فکری}

    \begin{moredi}
        \item تحلیل انتشار شایعه در شبکه‌های اجتماعی برخط با در نظر گرفتن مدل نظریهٔ بازی \cite{huang2020developing}
        \item ارائهٔ یک مدل شبیه‌سازی برای بررسی شرایط و نتیجه رسیدن به اجماع  در یک شبکهٔ برخط با دو گروه فکری مخالف با در نظر گرفتن کیفیت ارتباط‌ها \cite{bolzern2020opinion,nettasinghe2019diffusion}
        \item بررسی تأثیر اخبار انتشار بیماری کووید-۱۹ در شبکه‌های اجتماعی برخط \cite{cinelli2020covid}
    \end{moredi}

\end{frame}

\begin{frame}
    \frametitle{تغییرات آگاهی و رفتار اجتماعی}

    \begin{moredi}
        \item بررسی مدل بیماری \lr{SEIV}\LTRfootnote{Susceptible-Exposed-Infected-Vigilant} برای یک شبکه و تأثیر هوشیاری افراد بر تعداد ارتباط‌های فعال با دیگران و زمان رسیدن به حالت پایدار بدون بیماری \cite{li2019control,bhowmick2020influence}
        \item تأثیر آگاهی و میزان شیوع بیماری در ارتباط بین افراد در یک شبکهٔ دو لایه (یک لایه ثابت و یک لایهٔ متغیر)\cite{sahneh2019contact}
    \end{moredi}

\end{frame}

\section{پرسش های پژوهش}
\begin{frame}
    \frametitle{فرایند کلی حل مسئله}
    {
        \setbeamercovered{transparent}
        شبیه‌سازی:
        \begin{enumerate}\RTList
            \item<-1> تصادفی (محاسبهٔ وضعیت و شرایط جدید هر گره و به روز کردن همه گره‌ها در یک لحظه)
            \item<2-> آماری (محاسبهٔ امید ریاضی و میانگین وضعیت و شرایط انتقال برای کل شبکه در مدل مارکوف)

        \end{enumerate}

        مدل‌سازی:
        \begin{enumerate}\RTList
            \item تعریف متغیرهای فازی و توابع عضویت (فضای پیوسته)
            \item تعریف جدول قواعد فازی (ارتباط بین ورودی و خروجی‌های مسئله)
            \item تعریف روابط ریاضی تجمیع سازی برای هر گره
            \item<2-> تعریف مدل مارکوف معادل
            \item<2-> تعریف روابط آماری و کلی (مبتنی بر نظریهٔ میدان متوسط)
            \item<1> تعریف الگوی بیماری
            \item شبیه‌سازی

        \end{enumerate}
    }
\end{frame}

\section{روش و فنون پژوهش}
\subsection{بخش تئوری}
\subsection{بخش پیاده سازی}

\section{زمانبدی پیشنهادی }
\section{پیشینه پژوهش}

\section{منابع و مراجع}
\begin{frame}[allowframebreaks,noframenumbering]
    %    \frametitle{منابع}
    %    \nocite{*} % to add all reference in bib
    % \printbibliography
    \begin{latin}
        \bibliographystyle{unsrt-fa} % ieeetr-plain=unsrt
        \scriptsize
        \bibliography{./references}
    \end{latin}

\end{frame}


\begin{frame}[noframenumbering]
    \frametitle{نتیاج شبیه‌سازی آماری}

    \begin{figure}[htb]
        %	\vspace{-2cm}
        \centering
        \subfloat[یادگیری=$0.75$ و فراموشی=$0.5$]{\includegraphics[width=0.48\textwidth]{ode-75-50.pdf}}\
        \subfloat[یادگیری=$0.5$ و فراموشی $0.75$]{\includegraphics[width=0.48\textwidth]{ode-50-75.pdf}}
        \caption{نتیجهٔ اجرای شبیه‌سازی آماری در دو حالت}
        \label{fig:ode-sais}
    \end{figure}

\end{frame}

\end{document}



